\section{Conclusion and Recommendations}

The Digital Logic Simulator successfully addresses the limitations of traditional digital logic simulators by providing a lightweight, interactive, and user-friendly platform for designing, simulating, and analyzing digital circuits. By using C++ and the SFML graphics library, the project combines real-time simulation, intuitive graphical representation, and object-oriented design to create a modular system. Users can easily place logic gates, connect components, generate truth tables, and simplify Boolean expressions, all within a single integrated environment.

\vspace{0.5cm}
The project demonstrates the advantages of using object-oriented programming principles, such as encapsulation, and data abstraction, to structure the code effectively. Cross-platform support was achieved through SFML, making the simulator functional on both Windows and Linux. Overall, the Digital Logic Simulator provides an efficient, educational, and practical tool for students, hobbyists, and professionals working with digital logic circuits.

\vspace{0.5cm}
To further enhance the Digital Logic Simulator, additional features could be incorporated, such as more advanced gates and sequential components, improved Boolean expression parsing with automated K-map simplification, and the ability to save and load projects. Enhancements to the user interface, multiple inputs and outputs, addition of components like flipflops, MUX, DEMUX, etc. would improve usability. Expanding cross-platform support and integrating interactive tutorials could also make the simulator more accessible and educational for students and any user.
\clearpage