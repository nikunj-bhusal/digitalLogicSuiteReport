\section{Problems Faced and Solutions}
The Digital Logic Suite works well but has some limitations and ideas for future improvements. They are listed below as follows:

\begin{enumerate}
    \item Limitations
          \begin{itemize}
              \item \textbf{Gate Variety}: Only basic, universal and XOR gates are available; advanced components combinational and sequential circuits are missing.
              \item \textbf{No Circuit Persistence}: Currently, users cannot save or load their circuit designs for future editing or sharing.
              \item \textbf{Limited User Interface}: The interface is basic and may not be intuitive for all users, lacking features like drag-and-drop and right-click context menus.
              \item \textbf{No Error Checking}: The application does not provide feedback or warnings for invalid connections or logical errors in the circuit.
              \item \textbf{Limited Customization}: Users have minimal options to customize gate properties or circuit appearance which might reduce flexibility in design and presentation.
          \end{itemize}
    \item Future Enhancements
          \begin{itemize}
              \item \textbf{Advanced Components}: Add latches, flip-flops, registers and other logical components for more complex circuit simulations.
              \item \textbf{Save/Load Functionality}: Implement the ability to save, export, and import circuit designs.
              \item \textbf{Enhanced User Interface}: Improve usability with features like drag-and-drop, actual gate shapes instead of squares, and better visual feedback.
              \item \textbf{Error Detection}: Add real-time error checking and helpful messages for invalid circuit configurations.
              \item \textbf{Simulation Features}: Support step-by-step simulation, timing analysis, and visualization of signal propagation with the help of timing diagrams.
          \end{itemize}
\end{enumerate}

\clearpage
