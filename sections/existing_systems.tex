\section{Existing Systems}
Several tools for digital logic design are available, but they have limitations that our Digital Logic Simulator aims to address. Some of those systems are listed along with their limitations below:

\begin{itemize}
    \item \textbf{Logisim}: Logisim is a popular tool for learning how to design and simulate digital circuits, widely used by students and educators, but it doesn't work with modern C++. It is built in Java, which makes it slow on older or less powerful computers.
    \item \textbf{Digital Works}: This is a paid software that allows users to simulate basic logic gates and simple circuits. However, it lacks features like generating truth tables or simplifying Boolean expressions, which makes it less helpful for students.
    \item \textbf{Logic Friday}: Logic Friday is a free tool focused on simplifying Boolean expressions. It only works on Windows (does not have a cross-platform support), has no circuit simulation, and uses an old, hard-to-use interface..
    \item \textbf{Online Boolean Calculators}: Tools like CircuitVerse are web-based and allow users to design circuits and perform basic logic analysis online. However, they require a constant internet connection, can be slower than local applications, and may not offer the same performance as a C++-based tool like ours.
\end{itemize}
\clearpage
Key Improvements in Our System:
\begin{itemize}
    \item Our Digital Logic Simulator uses SFML to create a clear and responsive graphical interface, making it easy for users to place logic gates and build circuits. Our application runs smoothly even on less powerful computers.
    \item Our tool supports the full process of designing circuits, simulating them, generating truth tables, and simplifying Boolean expressions. This all-in-one approach makes it more useful for students.
    \item The Digital Logic Simulator works on multiple operating systems, including Windows, Linux, and macOS, ensuring that users can access it regardless of their device.
    \item Unlike web-based tools like CircuitVerse that depend on an internet connection, our app runs entirely offline which can make it more reliable and faster for users in areas with poor internet.
\end{itemize}
\clearpage