\section{Methodology}

This project, “Digital Logic Suite,” is a C++ application that uses the SFML graphics library and follows object-oriented programming (OOP) paradigm. We made different classes with private and public members to keep the code organized and safe. We used code reusability and data abstraction to make the project better. Each object has its own functions to handle events, and these work together to create the interactive simulation.

\vspace{0.5cm}
The Digital Logic Suite solves problems found in old digital logic simulators by giving users an easy-to-use tool made in C++ with SFML (Simple and Fast Multimedia Library). Unlike older tools that use only the command line, this simulator has a clear graphical interface. Users can see logic gates, circuit connections, and truth tables easily. By using OOP, we organized the code into classes, making it easier to reuse and maintain. Users can interact with the simulation, place components, and see how circuits work in real time. This makes it useful for learning and designing digital logic circuits.

\vspace{0.5cm}
Basic C++ graphics were not enough for our needs. To make a better and more interactive interface, we used SFML version 3.0.0. SFML gave us good graphics, worked on different systems, and it was easy to use for drawing, handling events, and getting input. We used VS Code as our IDE and the GCC compiler, which helped us to code, compile, and debug on different platforms.

\vspace{0.5cm}
Before starting, we learned SFML by reading tutorials, official documents, and online guides. We also reviewed OOP concepts to design a system that is easy to manage. SFML is used for features like making the grid layout, displying connections with wires, simulating logic gates, showing the component palette, drawing the window, creating buttons and the visual layout of the program.
\clearpage