\section{Limitations and Future Enhancements}
While the Digital Logic Simulator successfully implements the core functionalities, there are several areas that could be improved or added in the future to enhance usability and performance. One limitation encountered during the project was inconsistent code formatting among team members, which occasionally caused difficulties during Git merges and slowed down collaborative development. Also, due to time constraints, more desired features couldn't be added. Future enhancements that could be implemented include are listed below as follows:
\begin{itemize}
    \item \textbf{Extended Component Library}: Adding more advanced logic gates, multiplexers, decoders, and sequential circuit components.
    \item \textbf{Enhanced Sequential Simulation}: Support for complex sequential circuits with timing analysis and clocked operations.
    \item \textbf{Improved User Interface}: Features like drag-and-drop palettes, zooming, advanced grid snapping, and signal indicators to improve circuit visualization.
    \item \textbf{Cross-Platform Packaging}: Standalone installers for Windows, Linux, and macOS to make the tool more accessible.
    \item \textbf{Educational Features}: Integration of tutorials, guided exercises, or challenges to make the simulator more effective for learning purposes.
\end{itemize}

\vspace{0.5cm}
By addressing these limitations and implementing future enhancements, the Digital Logic Simulator can become a more robust, user-friendly, and educational tool for students, hobbyists, and professionals in the field of digital logic design.
\clearpage