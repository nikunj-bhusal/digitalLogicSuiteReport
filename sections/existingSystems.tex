\clearpage

\section{Existing Systems}

Many tools already exist for designing, analyzing, and simulating digital logic circuits. However, most of these tools focus on graphical user interfaces (GUIs), and they often lack support for command-line interaction, which is essential for automation and working on systems without graphics. Below are some of the most commonly used tools and their limitations:

\begin{itemize}
    \item \textbf{Logisim:} Logisim is a popular educational tool that provides a user-friendly drag-and-drop interface for creating and simulating digital circuits. It is excellent for beginners because it allows easy construction of circuits with logic gates and flip-flops. However, it does not support command-line use, cannot parse Boolean expressions directly, and lacks features for automating analysis or generating truth tables and Karnaugh maps through scripts.

    \item \textbf{Digital Works:} Digital Works allows users to build circuits with basic components like gates, flip-flops, and counters. It can perform simple simulations of circuits and show how signals change over time. But like Logisim, it is entirely GUI-based, with no command-line interface, and it does not support direct Boolean expression evaluation or advanced minimization features.

    \item \textbf{Logic Friday:} Logic Friday provides tools to simplify Boolean expressions, generate truth tables, and create Karnaugh maps. It is useful for analyzing equations but is limited to a Windows environment and GUI interface. There is no CLI support, and it does not offer circuit simulation or visualization outside its own interface.

    \item \textbf{Online Boolean Calculators:} Some websites provide online calculators to evaluate Boolean expressions or generate truth tables. While they can be convenient for quick tasks, they do not support advanced features like K-map minimization, simulation, or visualizing circuits. They also require internet access, which is not always available or reliable in every environment.
\end{itemize}

Despite the usefulness of these tools, there is a clear gap in the availability of a lightweight, cross-platform, command-line-based suite that combines Boolean expression parsing, truth table generation, K-map minimization, circuit simulation, and visualization. Our Digital Logic Suite aims to fill this gap by providing a single, modular CLI tool that allows users to perform all these tasks in one place, with the possibility of extending the suite in the future.
