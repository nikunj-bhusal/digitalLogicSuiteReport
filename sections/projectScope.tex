\clearpage
\section{Project Scope}

The Digital Logic Suite is mainly created to help students, teachers, and anyone interested in learning about digital logic circuits. It is made for academic and educational purposes, so it focuses on providing clear and useful tools to understand how digital logic works.

This project provides a command-line interface (CLI) tool, which means users can type commands in a terminal or console to use the program. This makes the tool lightweight and easy to run on many different computers without needing a complex graphical interface.

The suite is designed to be modular. This means it is built in separate parts or modules that can work independently but also fit together smoothly. Because of this modular design, it will be easier to add new features or improve existing ones in the future without needing to rewrite the entire system.

The main focus of the Digital Logic Suite is on combinational logic, which is a type of digital logic where the output depends only on the current inputs. The tool will provide features to analyze these logic circuits, simplify or minimize the logic expressions to make them easier to understand and use, and simulate how the logic behaves with different inputs.

Overall, this project aims to be a simple, effective, and flexible tool that supports learning and experimentation in digital logic.

